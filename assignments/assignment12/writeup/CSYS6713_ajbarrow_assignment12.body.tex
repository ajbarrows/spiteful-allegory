\textbf{Name:} \\

\medskip

\textbf{Conspirators:} 

\medskip
\medskip

\hrule

\medskip


\assignmentsonly{\pleasesubmitprojectdraft}

\begin{itemize}
\item 
  Please use Overleaf for writing up your project.
\item
  Build your paper using:
  \url{https://github.com/petersheridandodds/universal-paper-template}
\item
  Please use Github and Gitlab to share the code and data things you make.
\item
  For this first assignment, just getting the paper template up is enough.
\end{itemize}

\begin{enumerate}
  
\item

  Begin formulating project ideas.

  See storyology slides.
  
  General suggestion: Come up with some rich, text-based set of stories for analysis.

  For example: One (longish) book, or a book series, or a TV series.

  Data would be the original text (books), subtitles, screenplay, or scripts (TV series).

  \begin{itemize}
  \item 
    You must be able to obtain the full text.
  \item 
    You will want something with at least around $10^{5}$ words.
    More than $10^{6}$ would be great.
  \item 
    Transcripts of shows may be good for extracting
    temporal character interaction networks.
  \end{itemize}
  
  Please talk about possibilities with others in the class.
  
  For this assignment, simply list at least one possibility, noting the approximate text size
  in number of words, or whatever measure of size makes sense.

  
   \solutionstart

   %% solution goes here

   \solutionend

\item

  Lexical calculus:

  Derive the word shift equation for simple additive lexical instruments.

  You will have the derivation per class.

  The idea is to simply work through it yourself.

  There are no advanced mathematics here.

  But over and over, people do not understand what's going on.

  Word shifts are a kind of discrete derivative (difference) with words on the inside.

  Per lectures, the goal is to derive.
  $$
  \delta h_{\textnormal{avg},i}
  =
  \frac{100}{
    \left|
    \havgsup{(\rm comp)} - \havgsup{(\textnormal{ref})}
    \right|
  }
  \underbrace{
    \left[
      \havg{w_{i}} - \havgsup{(\textnormal{ref})}
      \right]
  }_{+/-}
  \underbrace{
    \left[
      p_{i}^{(\textnormal{comp})} - p_{i}^{(\textnormal{ref})}
      \right]
  }_{\uparrow/\downarrow}
  $$

  Performed in class and in numerous papers~\cite{dodds2009c,dodds2011e,dodds2015a}.

  
   \solutionstart

   %% solution goes here

   \solutionend

\end{enumerate}
