\textbf{Name:} \\

\medskip

\textbf{Conspirators:} 

\medskip
\medskip

\hrule

\medskip


\assignmentsonly{\pleasesubmitprojectdraft}

Prepare and deliver short presentations to introduce your project.

Instructions:

\begin{enumerate}
\item 
  All students must attend in person unless enrolled as online only
  (reminder: this is true of all lectures, barring reasonable excuses).
  
\item 
  Talks must be G rated and respectful of others.
  See the
  \wordwikilink{https://pdodds.w3.uvm.edu/teaching/courses/2023-2024pocsverse/docs/2023-2024pocsverse-syllabus.pdf}{PoCS syllabus},
  %% can't use perl for overleaf
  %% fix this:
  %% \syllabuslink{\coursewebsite/docs/\courseprefix},
  UVM's student conduct standards,
  and UVM's
  \wordwikilink{https://www.uvm.edu/president/our-common-ground}{Our Common Ground}.

\item
  Time: Please aim for no more than 5 minutes per person.

\item
  For your talk, your mission is to
  give an overview of your project:
  \begin{enumerate}
  \item
    Explain what you are exploring and why it matters.
  \item
    Give an overview of data (origin, curation, quality, other issues).
  \item
    Optionally describe any progress so far.
  \item
    Outline what you're hoping to achieve.
  \end{enumerate}

\item
  At the start, please introduce yourself in a sentence (name + your department/field),
  and to acknowledge who you're working with.

\item
  Slides: Suggest 3 to 5.  More may work but
  \wordwikilink{https://www.youtube.com/watch?v=xOrgLj9lOwk}{100 is right out}.
  Quality of slides forms part of the grade.

\item
  Slide format: PDFs are the least dangerous.
  Beamer/LaTeX is encouraged but not required.
  Powerpoint or Keynote will be fine as well. Probably.

\item
  Students who are online only may share their screen if needed.

\item

  \textbf{Please upload your slides to Teams some time on Wednedsay, April 3, the day before the talk session on Thursday, April 4}.
  Use the channel ``Projects---slides, videos, reports.''

\item
  Practice!  These are short talks so you can run through
  them a number of times to straighten everything out.


\end{enumerate}
