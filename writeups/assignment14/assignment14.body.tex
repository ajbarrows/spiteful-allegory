\textbf{Name: Anthony Barrows} \\

\medskip

\textbf{Conspirators:} 

\medskip
\medskip

\hrule

\medskip


%% pride and prejudice
%% frankenstein
%% moby dick

%% share the data needed here with:
%% 
%% assignment13-share-story-data.sh

\assignmentsonly{\pleasesubmitprojectdraft}

\assignmentsonly{Semester goal: A paper per text studied, building through assignments.}

%% pride and prejudice

\textbf{Three stories to analyze:}

\begin{itemize}
\item
  \textbf{Pride and Prejudice}\\
  \url{https://www.gutenberg.org/ebooks/1342}
\item
  \textbf{Frankenstein; or the Modern Prometheus}\\
  \url{https://www.gutenberg.org/files/84/84-h/84-h.htm}
\item
  \textbf{Moby Dick; Or, The Whale}\\
  \url{https://www.gutenberg.org/ebooks/2701}
\end{itemize}

\textbf{Data:}

For this assignment, the novels have been processed into 1-grams
with an attempt to capture all elements including punctuation.

You can use the data below, or what you produced in the previous assignment.

The basic data format is as a time series with one 1-gram per line (links below).

For each story, also linked to below are the rank distributions of 1-grams by counts.

\url{https://pdodds.w3.uvm.edu/permanent-share/pride_and_prejudice_narrativetimeseries.txt}\\
\url{https://pdodds.w3.uvm.edu/permanent-share/pride_and_prejudice_1grams.txt}

\url{https://pdodds.w3.uvm.edu/permanent-share/frankenstein_narrativetimeseries.txt}\\
\url{https://pdodds.w3.uvm.edu/permanent-share/frankenstein_1grams.txt}

\url{https://pdodds.w3.uvm.edu/permanent-share/moby-dick_narrativetimeseries.txt}\\
\url{https://pdodds.w3.uvm.edu/permanent-share/moby-dick_1grams.txt}

\textbf{Instrument---Shifterator:}

\begin{itemize}
\item 
  For word shifts,
  use the Python package described in Ref.~\cite{gallagher2021a}.

  Links to paper versions (arXiv is always best),
  Github repository,
  and an
  exhilarating Twitter feed
  can be found here: \url{https://pdodds.w3.uvm.edu/research/papers/gallagher2021a/}.

  (Various Matlab versions made by the Unreliable Deliverator do exist
  and need to be shared on Gitplaces.
 A more sophisticated map+list version has long been in development.)

  
  Github repository:
  \url{https://github.com/ryanjgallagher/shifterator}
\end{itemize}

\fbox{\begin{minipage}{40em}
		Code  is available at \href{https://github.com/ajbarrows/spiteful-allegory}{https://github.com/ajbarrows/spiteful-allegory}.
		
		See \texttt{/work/notebooks}
\end{minipage}}


\begin{enumerate}

\item

  Measure the average happiness of each text
  using the labMT word list
  with the lexical lens:
  \begin{equation}
%    \typelens
	\mathcal{L}
    =
    \left\{
    \elementsymbol
    \in
    \systemsymbol
    \|
    \havg{\elementsymbol} \le 4
    \
    \mbox{or}
    \
    \havg{\elementsymbol} \ge 6
    \right\}
    \label{eq:lexicallens}
  \end{equation}
  where
  $\systemsymbol$ is the labMT lexicon
  and
  $\elementsymbol$ is a word in $\systemsymbol$.

\solutionstart

\begin{figure}[H]
	\centering
	\includegraphics[width=0.5\textwidth]{08_reporting/assign14_trunc_labMT_dist.png}
	\label{fig:happiness_dist}
	\caption{Happiness distribution with $\mathcal{L}$ removed.}
\end{figure}

\begin{tabular}{lll}
	Text & $h_{avg}$ & $ h_{avg}$ weighted by word count \\
	\hline
	\textit{Pride and Prejudice} & 5.74 & 5.99 \\
	\textit{Frankenstein} & 5.64 & 5.85 \\
	\textit{Moby Dick} & 5.67 & 5.77
\end{tabular}

\solutionend

\item (3 points each)

  For the following 
  $\Tref$
  and
  $\Tcomp$
  for each novel,
  generate a collection of word shifts
  as described below.

%  Continue to use the same lexical lens $\typelens$ as above.
  Continue to use the same lexical lens $\mathcal{L}$ as above.

  Pride and Prejudice:


  \begin{itemize}
  \item
    Pride and Prejudice:\\
    $\Tref$ = the first 40\% of the book,\\
    $\Tcomp$ = 70\% to 75\% of the book.
  \item 
    Frankenstein:\\
    $\Tref$ =  the first 20\% of the book,\\
    $\Tcomp$ = the last 10\% of the book.
  \item
    Moby Dick:\\
    $\Tref$ =  the whole book,\\
    $\Tcomp$ = 80\% to 90\% of the book.
  \end{itemize}


  \begin{enumerate} 
  \item 
    For each novel,
    produce word shifts comparing text $\Tcomp$ relative to text $\Tref$.
    Important: Use the average happiness of text $\Tref$ as the baseline (this is not the default in the Python package).

  \item 
    Interpret the word shifts. Does what you see make sense?
    Are there any surprises?
    Are some words being used in what the average person might not think is their primary meaning?
    For example, ``crying'' in Moby Dick means yelling, and ``sick'' can mean ``awesome.''
  \item
    Reverse the comparison:
    Produce word shifts comparing text $\Tref$ relative to text $\Tcomp$.
    Now use the average happiness of text $\Tcomp$ as the baseline.
    
  \item
    Comment on any asymmetries you see (the basic word shifts we use are asymmetric).
  \item
    Produce word shifts comparing text $\Tref$ relative to text $\Tcomp$.
    Now use 5 as the baseline reference score (neutral on the happiness-sadness spectrum of 1--9 for labMT).
    
  \item
    Compared to your first word shifts, how interpretable are these ones?
  \end{enumerate}

  
   \solutionstart
	
	\begin{enumerate}
		\item 
		Happiness word shifts using average happiness of $T_{ref}$ as the baseline. I left off the cumulative and text size decorators. It would require a bit of code modification to get them in a subplot.
		
			\begin{figure}[H]
				\centering
				\includegraphics[width=\textwidth]{08_reporting/assign14a_wordshift}
				\label{fig:wordshift}
			\end{figure}
	
		\item
		Each of these comparison texts happens to be the part of each respective book when the story's going poorly for the main character. Elizabeth's rejected Mr. Darcy, Mr. Darcy messed up a marriage proposal between Elizabeth's sister Jane, Elizabeth knows she felt strongly about Mr. Darcy, but he's \textit{gone}. Less talk of pleasure and happiness. More talk of family, but that's just what the story's about. The first 20\% of \textit{Frankenstein} is mostly Captain Walton's exploration of the North Pole and encounters with Victor Frankenstein. The last 10\% is probably when this expedition ends horribly, including the captain's speech about what miserable things the crew agreed to participate in. In \textit{Moby Dick}, Ishmael ends up floating in the ocean. 
		
		Selecting parts of texts known to be especially contrasting allows the word shift plots to pick up on large changes in sentiment easily. It would be tricky to tease out negation: `friends,' `sun,' `happiness,' and other positive words appear in a distinctly negative portion of \textit{Frankenstein}, suggesting those words might be preceded by `no,' or a similar negative context. 
		
		
		\item 
	
			\begin{figure}[H]
	  				\centering
	  				\includegraphics[width=\textwidth]{08_reporting/assign14c_wordshift}
	  				\label{fig:wordshift_rev}
	  			\end{figure}
	
		
		\item 
		
		The scores are approximately the same, but positive words shift around in their ranking. `Hope' climbs above `killed' in \textit{Pride and Prejudice}, `me' above `miserable' in \textit{Frankenstein}, and `happiness' above `brutal' in \textit{Moby Dick}.
		
		\item 
		Word shifts using 5 as the baseline reference score.
		
	      	\begin{figure}[H]
		       	\centering
		       	\includegraphics[width=\textwidth]{08_reporting/assign14e_wordshift}
		       	\label{fig:wordshift_ref}
	    	\end{figure}
	
		
		\item 
		Using a `neutral' reference score paints an entirely different picture, one from which you probably couldn't draw solid conclusions. With a neutral score, words are effectively evenly distributed, such that words used more (or less) frequently contribute equally, regardless of whether they're positive or negative in happiness score. We know this doesn't make sense. A more direct comparison has to be with reference to the comparison text's sentiment score -- this makes sense, now. Happiness is a relative thing; nothing (except chocolate) is `objectively' happy, perhaps.
   
		
	

	\end{enumerate}
   %% solution goes here

   \solutionend

\end{enumerate}

