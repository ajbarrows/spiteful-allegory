\textbf{Name: Anthony Barrows} \\

\medskip

\textbf{Conspirators:} 

\medskip
\medskip

\hrule

\medskip


\assignmentsonly{\pleasesubmitprojectdraft}

\begin{itemize}
\item 
  Please use Overleaf for writing up your project.
\item
  Build your paper using:
  \url{https://github.com/petersheridandodds/universal-paper-template}
\item
  Please use Github and Gitlab to share the code and data things you make.
\item
  For this first assignment, just getting the paper template up is enough.
\end{itemize}

\fbox{\begin{minipage}{40em}
		Code (if applicable) is available at \href{https://github.com/ajbarrows/spiteful-allegory}{https://github.com/ajbarrows/spiteful-allegory}.
\end{minipage}}
\begin{enumerate}
  
\item

  Begin formulating project ideas.

  See storyology slides.
  
  General suggestion: Come up with some rich, text-based set of stories for analysis.

  For example: One (longish) book, or a book series, or a TV series.

  Data would be the original text (books), subtitles, screenplay, or scripts (TV series).

  \begin{itemize}
  \item 
    You must be able to obtain the full text.
  \item 
    You will want something with at least around $10^{5}$ words.
    More than $10^{6}$ would be great.
  \item 
    Transcripts of shows may be good for extracting
    temporal character interaction networks.
  \end{itemize}
  
  Please talk about possibilities with others in the class.
  
  For this assignment, simply list at least one possibility, noting the approximate text size
  in number of words, or whatever measure of size makes sense.

  
   \solutionstart

   %% solution goes here

For whatever reason, the first thing that comes to mind that would be fun to analyze is a play. Character interactions and line delivery are critical; plays are horrendously boring if interactions are poorly displayed. I'm not sure if these texts will be long enough.
	
Some plays:
\begin{itemize}
	\item Samuel Beckett's Waiting for Godot is $10^4$
	\item Authur Miller's Death of a Salesman is $10^4$
	\item Even Shakespeare's Hamlet, known for being long, is $10^4$
\end{itemize}

That might not work. Full seasons of TV shows seem to have the same problem. Perhaps there's merit in combining some.

Books are a bit more straightforward since, thanks to many large-scale projects, curated texts seem to be readily available. For example:

\begin{itemize}
	\item Notoriously long, Tolstoy's \textit{War and Peace}: $10^5$
	\item Hugo's \textit{Les Miserables}: also $10^5$
	\item Powell's 	\textit{A Dance to the Music of Time}: $10^6$!
\end{itemize}

There are options. It would be helpful to know more about what we plan to do with the text, I suppose.

$\Rightarrow$ During class today, you mentioned \texttt{convokit}. Very cool. I had a look. The dataset of surpreme court listings would be fun.
	
	\begin{lstlisting}[language=python]
		Number of Speakers: 8979
		Number of Utterances: 1700789
		Number of Conversations: 7817
	\end{lstlisting}
	
	That should do it. 
	
	
   \solutionend

\item

  Lexical calculus:

  Derive the word shift equation for simple additive lexical instruments.

  You will have the derivation per class.

  The idea is to simply work through it yourself.

  There are no advanced mathematics here.

  But over and over, people do not understand what's going on.

  Word shifts are a kind of discrete derivative (difference) with words on the inside.

  Per lectures, the goal is to derive.
  $$
  \delta h_{\textnormal{avg},i}
  =
  \frac{100}{
    \left|
    \havgsup{(\rm comp)} - \havgsup{(\textnormal{ref})}
    \right|
  }
  \underbrace{
    \left[
      \havg{w_{i}} - \havgsup{(\textnormal{ref})}
      \right]
  }_{+/-}
  \underbrace{
    \left[
      p_{i}^{(\textnormal{comp})} - p_{i}^{(\textnormal{ref})}
      \right]
  }_{\uparrow/\downarrow}
  $$

  Performed in class and in numerous papers~\cite{dodds2009c,dodds2011e,dodds2015a}.

  
   \solutionstart

   %% solution goes here
   
$\delta h_{\textnormal{avg},i}$ is the change in happiness, $h$, on average, for word $i$ (i.e., the contribution from word $i$), given two texts (reference and comparison). The expression above derives from the measure of happiness of a given text, as the sum of the happiness associated with all words $i$, weighted by each word's fractional abundance in the text (probability):

\[
\theta_{\textnormal{avg}}
=
\sum_{i=1}^{N}
p_i\theta_i
\]


To quantify a difference, we would simply consider the differences in happiness for each text (with and without word $i$):

\[
\begin{gathered}
	h^{\textnormal{comp}}_{\textnormal{avg}}
	-
	h^{\textnormal{ref}}_{\textnormal{avg}}
	=
	\sum_{i=1}^{N}p_i^{\textnormal{comp}}h_{\textnormal{avg}}(w_i)
	-
	\sum_{i=1}^{N}p_i^{\textnormal{ref}}h_{\textnormal{avg}}(w_i) \\
	=
	\sum_{i=1}^{N} (p_i^{\textnormal{comp}} - p_i^{\textnormal{ref}} )h_{\textnormal{avg}}(w_i) \\
	=
	\sum_{i=1}^{N} (p_i^{\textnormal{comp}} - p_i^{\textnormal{ref}} ) (h_{\textnormal{avg}}(w_i) - h^{\textnormal{ref}}_{\textnormal{avg}} + h^{\textnormal{ref}}_{\textnormal{avg}}) \\
	=
	\sum_{i=1}^{N} (p_i^{\textnormal{comp}} - p_i^{\textnormal{ref}} ) (h_{\textnormal{avg}}(w_i) - h^{\textnormal{ref}}_{\textnormal{avg}}) + 
	\underbrace{
		\sum_{i=1}^{N} (p_i^{\textnormal{comp}} - p_i^{\textnormal{ref}} ) h^{\textnormal{ref}}_{\textnormal{avg}}
	}_{0}
\end{gathered}
\]

And so,
\[
  \delta h_{\textnormal{avg},i}
=
(h^{\textnormal{comp}}_{\textnormal{avg}} - h^{\textnormal{ref}}_{\textnormal{avg}})
(h_{\textnormal{avg}}(w_i) - h^{\textnormal{ref}}_{\textnormal{avg}})
(p_i^{\textnormal{comp}} - p_i^{\textnormal{ref}} )
\]







   \solutionend

\end{enumerate}
