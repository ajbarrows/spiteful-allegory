\textbf{Name:} \\

\medskip

\textbf{Conspirators:} 

\medskip
\medskip

\hrule

\medskip


\assignmentsonly{\pleasesubmitprojectdraft}

\assignmentsonly{Goal: A paper per text studied, building through assignments.}

%% - Characters: ``x-ray''
%% 
%% - punctuation
%% 
%% Assignments:
%% - [ ] Emotion words in LLM ...
%% 
%% Read this paper:
%% - [ ] Semantic differentials in LLMs ...
%% 
%% Write up slides for word shifts (appendix from 2015)

\begin{enumerate}

\item (3 points each)

  Using your text of choice, generate word shifts comparing two ``interesting'' regions of text.

  Use the Python package described in Ref.~\cite{gallagher2021a}.

  (Various Matlab versions made by the Unreliable Deliverator do exist and need to be shared on Gitplaces. A more sophisticated map+list version has long been in development.)

  Links to paper versions (arXiv is always best),
  Github repository,
  and an
  exhilarating Twitter feed
  can be found here: \url{https://pdodds.w3.uvm.edu/research/papers/gallagher2021a/}.

  ``Interesting'' is anything you find interesting.  Could be books 3 and 12 in a series,
  second half of a book compared to the first half, season 4 of a show versus all seasons, etc.

  Aim to find two texts that are both reasonably large (more than $10^{4}$ words)
  and fairly different in average happiness scores (though even the same scores can be
  meaningfully explored with word shifts).
  
  Let's call the two texts
  $\texta$
  and
  $\textb$.
  In your plots, you should label them meaningfully based on your choices).

  Use a reasonable exclusion lens of your choice, e.g., [4, 6] or [3, 7].

  \begin{enumerate} 
  \item 
    Produce a word shift comparing text $\textb$ relative to text $\texta$.
    Use the average happiness of text $\texta$ as the baseline.
  \item 
    Interpret the word shift. Does what you see make sense?
    Are there any surprises?
    Are some words being used in what the average person might not think is their primary meaning?
    For example, ``crying'' in Moby Dick means yelling, and ``sick'' can mean ``awesome.''
  \item
    Produce a word shift comparing text $\texta$ relative to text $\textb$.
    Use the average happiness of text $\textb$ as the baseline.
  \item
    Comment on any asymmetries you see (the basic word shifts we use are asymmetric).
  \item
    Produce a word shift comparing text $\texta$ relative to text $\textb$.
    Now use 5 as the baseline reference score (neutral on the happiness-sadness spectrum of 1--9).
  \item
    Compared to your first word shift, how interpretable is this one?
  \end{enumerate}

  
   \solutionstart

   %% solution goes here

   \solutionend

\end{enumerate}

