%% replace 'papertag' with short distinct tag
%% - helps with distinction for inclusion in other documents
%% - always

%% procedure for figures:
%% adding directory figures/localized/
%% before figures in includegraphics
%% commands forces search and localize

\section{Introduction}
\label{sec:papertag.introduction}


The field of neuroimaging suffers from a crisis of reproducibility, best documented in \cite{MarekEtAl2022}, which demonstrated that resolving individual differences when effect sizes are small often requires thousands of participants. Pioneers in the field have built fantastic tools like \href{https://neurosynth.org/}{\texttt{Neurosynth}} \cite{YarkoniEtAl2011} to extract relevant information from each paper in their vast database of neuroimaging studies. By the authors' own admission, this technique is crude, and risks erroneously connecting psychological constructs by identifying terms (e.g., attention, ADHD, depression) and functional magnetic resonance imaging (fMRI) image coordinates that appear in the same paper. This approach is unprecedented and could lead to transformative work, but it also presents several key drawbacks:

\begin{itemize}
	\item Methods for processing analyzing MRI scans are far from harmonized.
	\item \cite{owensRecalibratingExpectationsEffect2021} showed that even \textit{with} large-scale imaging efforts effect sizes remain small, suggesting that under-powered studies reporting large effects may involve spurious associations
	\item Neurosynth relies on word associations that may not represent psychological constructs, and fails to assign meaning to brain map coordinates (e.g., ``activation'' vs. ``deactivation''). 
	\item For various reasons (some more principled than others), not all papers report coordinates.
\end{itemize}

\begin{figure}[tp!]
  \centering	
    \includegraphics[width=\columnwidth]{figures/localized/null.pdf}  
  \caption{
    \todo{Add caption}
  }
  \label{fig:papertag.}
\end{figure}

\section{Description of data sets}
\label{sec:papertag.data}

\subsection{Text}
\todo{Describe Neurosynth and Neuroquery}

Abstracts for all texts contained in the most recent Neurosynth and Neuroquery databases were downloaded using NiMARE, a Python package meant for neuroimaging study meta-analysis \cite{SaloEtAl2022, SaloEtAl2023}. Combining abstracts from these sources resulted in texts from 19,148 publications, uniquely identified by their PubMed ID (PMID). 

\section{Model}
\label{sec:papertag.model}

\section{Results}
\label{sec:papertag.results}

\todo{Explain what we found}

\section{Concluding remarks}
\label{sec:papertag.concludingremarks}

\todo{Bring it home.}

\section{Methods}
\label{sec:papertag.methods}

All code is available at \href{https://github.com/ajbarrows/spiteful-allegory}{https://github.com/ajbarrows/spiteful-allegory}.

\subsection{Proposed Approach}

Instead of focusing solely on results (i.e., brain-behavior associations), I propose to focus on common methods. \textbf{Goal: Use existing literature to produce a basic description of neuroimaging analysis strategies used to model various brain-behavior relationships.} 

If successful, this approach could:
\begin{itemize}
	\item Inform which analytical strategies (e.g., pre-processing, variable and model selection, external validity estimates) are most often used, and under which conditions
	\item Identify which strategies are most effective (\textit{or at least lay the groundwork})
\end{itemize}


\todo{Add methods.}


\section{Scratch}
\begin{figure}
	\includegraphics[width=\columnwidth]{figures/project/abstract_rankcount}
	\caption{Rank-count distribution for word frequencies in all Neurosynth and Neuroquery abstracts.}
\end{figure}

